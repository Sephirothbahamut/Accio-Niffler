%TODO
\paragraph{Exploration mode}

The player is free to move around  the map in both human form that cat form. If Delphini is also in the player's party, she will follow the player in his every move, but if he is in cat form and enters a place not accessible to humans, Delphini will remain waiting for him When the player is faced with a obstacle, mysterious object or secret passage, he can: 

\begin{itemize}
    \item Use the spells that he has learning to solve the puzzle.
    \item Has the possibility of being able to use the transfiguration to transform small objects into somethings that is needed at the moment.
    \item Transform into cat form and reach places inaccessible to human.
\end{itemize}

The player can interact with NPCs:

\begin{itemize}
    \item In human form he can talk to them, they can make funny story or riddles.
    \item In cat form he can be stroked.
\end{itemize}

\fullwidthgraphicscaption{../Pictures/Gameplay/Dialogue_picture.jpg}{Game: Doom\&Destiny}

%TODO
\paragraph{Combat mode}

Il sistema di combattimento è a turni con un solo cobattente autorizzato a compiere un'azione alla volta. All'inizio del combattimento personaggi e mostri vengono aggiunti alla scena e poi in base alle proprie statistiche viene scelto l'ordine di attacco. Quando nel party è presente anche Delphini il player può decidere solo l'azione di questo NPC, ovvero se attaccare oppure farle usare uno strumento. Quando il player decide l'azione di attacco per Delphini possono esserci due casi:

- Se il livello di amicizia è basso, l'attacco di Delphini è randomico.
- Se il livello di amicizia è alto, Delphini userà una spell per l'attacco combinato.

Il combattimento inoltre è in stile D\&D, ma le regole sono state leggermente modificate per rendere il gioco fluibile e divertente.

\fullwidthgraphicscaption{../Pictures/Gameplay/Combat_picture.png}{Game: Octopath traveler}
\pagebreak

\paragraph{Player rewards}

The player's reward are obtainable in every fight. In the specific combat with Cerberus the player can get: 60 Galleon, 80 Sickle and 100 Knut

\centeredgraphicssize{../Pictures/Gameplay/Rewards_picture.jpg}{5cm}

\paragraph{ Experience points and friendship experience}

Characters gain Experience Points by completing challenges, missions and minigames (lessons). When Experience Points reach a certain treshold, the Character's level rises. 
Each level grants all the spells of that level which have already been seen during a lesson.

Delphini has an stat in Frienship Points. These increase and decrease depending on the player's choices and successes in missions involving Delphini.

%TODO *describe how delphini acts depending on friendship points*

\paragraph{Gryffindor points}

When the player completes a task really well (perfect score in minigames, extremely good timing in timed tasks etcc) or when he fails terribly, some points are added to or removed from the player's House. Sometimes player choices can affect the points of other houses as well.
Grants an achievement on the chosen game platform.

\paragraph{Checkpoints }

Checkpoints are save points where the player can go back after a defeat. They appear as a Niffler statue which activates when the player makes an offer with a special coine called "Niffler Galleon".
\centeredgraphicssize{../Pictures/Gameplay/Statue_niffler_picture.jpg}{5cm}

\paragraph{Saving}

Checkpoints also save the game. The player can still save at anytime through the main menu. During battles saving is disabled.
\pagebreak