\section{File Naming Convention}
Generally, each file starts with capital letter; spaces are replaced by underscores.
The filename represents its content in roughly two parts:

		\begin{itemize}
			\item The first part represents what is contained in the file or what it refers to (for example: Albus\_Dumbledore represents an asset concerning the character Albus Dumbledore, Black\_Lake\_ represents an asset concerning the Black Lake setting and so on). It is composed by no more than 3 words, separated by underscores.
			\item The second part represents what type of content the file belongs to (for example: \_texture,\_sound, \_map, \_circumplex). It must be composed by only one word and it must be the final one of the filename: for this reason, abbreviations are allowed (for example: relationship map can be shortened to \_relmap) 
	
			\end{itemize}

List of types to identify files:
		\begin{itemize}
			\item \_source
			\item \_diagram
			\item \_flowchart
			\item \_template
			\item \_circumplex
			\item \_portrait
			\item \_relmap
			\item \_map
			\item \_picture
			\item \_document
			\item \_logo
			\item \_sheet
			\item \_icon
			\item \_chart
		\end{itemize}

Each diagram or flowchart must be paired with its source that generated it. \\

E.g.

\begin{itemize}
			\item Albus\_circumplex.png represents the character circumplex of Albus Dumbledore
			\item Myrtle\_Ghost\_portrait.png represents a portrait image of Myrtle in her ghostly form
			\item Minerva\_after\_event\_relmap.png represents the relationship map of Minerva McGonagall after a traumatic event that happens through-out the story
\end{itemize}

Exported documents are exempt from this naming convention and will instead follow this one:

[Document Abbreviation]\_[Name of the team]

For example the abbreviation of the export of the Data Organization Document (DOD) will be:

\begin{itemize}
			\item DOD\_AccioNiffler.docx
			\item DOD\_AccioNiffler.pdf
\end{itemize}

\pagebreak