\section{File Naming Convention}
Generally, each file starts with capital letter; spaces are replaced by underscores.
The filename represents its content in roughly two parts:

\begin{itemize}
	\item The first part represents what is contained in the file or what it refers to (for example: Albus\_Dumbledore represents an asset concerning the character Albus Dumbledore, Black\_Lake\_ represents an asset concerning the Black Lake setting and so on). It is composed by no more than 3 words, separated by underscores.
	\item The second part represents what type of content the file belongs to (for example: \_texture,\_sound, \_map, \_circumplex). It must be composed by only one word and it must be the final one of the filename: for this reason, abbreviations are allowed (for example: relationship map can be shortened to \_relmap).
\end{itemize}

List of types to identify files in artistic assets:
\begin{itemize}
	\item \textbf{\_image} A full, independant image (portarits, backgrounds, etcc).
	\item \textbf{\_music} Background music.
	\item \textbf{\_sound} A sound effect.
	\item \textbf{\_source} The modifiable file format (For example a .pdn for images, a .mid/.midi for music etcc).
	\item \textbf{\_tile} An individual tile.
	\item \textbf{\_tileset} An image intended to be read in tiles (environment tilesets, icon sets, face sets, etcc).
	\item \textbf{\_video}
	\item \textbf{\_voice} Spoken dialogue or part of a dialogue.
\end{itemize}

List of types to identify files in documentation:
\begin{itemize}
	\item \textbf{\_source} The modifiable file format (For example a .pdn for images, a .mid/.midi for music etcc).
	\item \textbf{\_diagram}
	\item \textbf{\_flowchart}
	\item \textbf{\_template}
	\item \textbf{\_circumplex}
	\item \textbf{\_portrait}
	\item \textbf{\_relmap}
	\item \textbf{\_map}
	\item \textbf{\_picture}
	\item \textbf{\_document}
	\item \textbf{\_logo}
	\item \textbf{\_sheet}
	\item \textbf{\_icon}
	\item \textbf{\_chart}
\end{itemize}

Each diagram or flowchart must be paired with its source that generated it. \\

E.g.

\begin{itemize}
	\item Albus\_circumplex.png represents the character circumplex of Albus Dumbledore.
	\item Myrtle\_Ghost\_portrait.png represents a portrait image of Myrtle in her ghostly form.
	\item Minerva\_relmap.png represents the relationship map of Minerva McGonagall.
	\item Minerva\_relmap\_source.pdn the editable source for the previous file.
\end{itemize}

\subsubsection*{Exceptions}

Documentation .tex source files mustn't end in \textit{\_source}. Except for the root file, all the other files must have the same name as the title of the section they contain. If a file is further split in other sub-files, all those they must be inside of a folder named the same as the parent file. All those files and folders still have the spaces replaced with underscores.

Final .pdf documentation files exported from the LaTeX source are named as follows:

[Document Abbreviation]\_[Name of the team]

For example the abbreviation of the export of the Data Organization Document (DOD) will be \textbf{DOD\_AccioNiffler.pdf}

\pagebreak