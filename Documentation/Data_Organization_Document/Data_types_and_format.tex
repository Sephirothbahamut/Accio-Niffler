\section{Diagrams}
Here is a comprehensive list of software to be used in this project, along with their version on which they should be used to reduce the possibility of compatibility issues.
\paragraph{.drawio}
Save format for the draw.io software. It works as a source for diagrams, allowing for easy and quick edits when needed. Diagrams are then exported in the .png format
\paragraph{.png}
The image exported from draw.io files to be inserted in the documentation.

\section{Text}
\paragraph{.tex}
LaTeX file extension

\section{Pictures}
Here is a comprehensive list of software to be used in this project, along with their version on which they should be used to reduce the possibility of compatibility issues.
\paragraph{.piskel} Piskel project format that keeps all the editing information.
\paragraph{.pdn} Paint.net format that keeps all the editing information.
\paragraph{.jpg and .png (documentation)} tandard format for images with no constraints (for example for documentation pictures like characters' portrait, settings and so on...)
\paragraph{.drawio} .png (game) \\

Exported image for game usage.
\begin{itemize}
	\item All images have a 32 bits color depth.
	\item Background images: 1920*1080
	\item Tilesets: each tile is 64*64 pixels
	\item Iconset: each tile is 64*64 pixels.
	\item Faceset: each tile is 256*256. Each file must contain different facial expressions for the same character.
	\item  Character set: each tile is 64*64 pixels. A character set must be 60 tiles wide and at most 4 tiles tall.
		\begin{itemize}
			\item Each 60 horizontally consecutive tiles form a contiguous walking animation, with the 2nd one representing the idle state. 
			\item Each row represent an animation; non-rotating objects only have one row
			\item Objects with an animation for each direction have 4 rows in the following order: right, up, left, down
		\end{itemize}
\end{itemize}


\section{Video }
\paragraph{.mkv, .mp4}
		\begin{itemize}
			\item Resolution: 1920*1080
			\item FPS: 60
			\item Audio Sample Rate: 48000 Hz
			\item Audio Channels: Stereo
		\end{itemize}


\section{Audio }
\paragraph{.ogg Vorbis Audio File}
		\begin{itemize}
			\item Audio Channels: Stereo
			\item Sample Rate: 44100 Hz
			\item Nominal Bitrate: 64 kbit/s
		\end{itemize}

\section{Game data}
\paragraph{.rpgproject}
		 Main project file for RPG Maker MV software 

\paragraph{.js}
	 For scripting purposes inside the RPG Maker MV software

\paragraph{.json}
	 For storing various asset information inside the RPG Maker MV software

\section{Unity}

\paragraph{.prefab} Unity’s Prefab system allows you to create, configure, and store a GameObject complete with all its components, property values, and child GameObjects
 as a reusable Asset.
\paragraph{.scene} Scenes contain the environments and menus of your game.
\paragraph{.mat} In Unity, you use materials and Unity shaders together to define the appearance of your scene
\paragraph{.asset}A Unity asset is an item that you can use in your game or Project. An asset may come from a file created outside of Unity, such as a 3D model, an audio file, an image, or any of the other types of file that Unity supports. There are also some asset types that you can create within Unity, such as an Animator Controller, an Audio Mixer or a Render Texture.
\paragraph{.cs} Contains the code









\pagebreak