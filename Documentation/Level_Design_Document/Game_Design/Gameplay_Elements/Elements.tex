
\subsubsection{Exploration Mode}

The player is free to move around  the map in both human form that cat form. If Delphini is also in the player's party, she will follow the player in his every move, but if Minerva is in cat form and enters a place not accessible to humans, Delphini will remain waiting for Minerva. 

When the player is facing an obstacle, mysterious object or secret passage, he can: 

\begin{itemize}
    \item Use the spells that he has learnt to solve the puzzle.
    \item Use transfiguration to transform small objects into somethings that is needed at the moment.
    \item Transform into cat form and reach places inaccessible to a human.
\end{itemize}

The player can interact with NPCs:

\begin{itemize}
    \item With Minerva in human form she can talk to them, they can reveal details about the plot, joke, get entertained with riddles.
    \item With Minerva in cat form she can be stroked.
\end{itemize}

\fullwidthgraphicscaption{../Pictures/Gameplay/Dialogue_picture.jpg}{Game: Doom\&Destiny}

\pagebreak 

\subsubsection{Combat System}
The combat is turn based. It mostly follows D\&D rules, only exception being that space and movement don't play any role: the combat happens like an old school sideview turn-based rpg. 

All the D\&D rules that involve moving around don't play any role. Area spells and skills will hit multiple enemies without checking an on-map position. Finally maximum range values for ranged spells and attacks will not play any role. 

All this is counterbalanced by the lack of a more conventional D\&D-sized party, since the player is mostly alone or with a single party member, and will be facing up to 3 opponents per battle.

The main reason is increasing the pace of the fight, since a more complex system which involves movement, getting closer to the turn-based strategy games paradigm, would be too slow paced and complex for the ideal target audience, and is better suited for games where a battle involves 5+ units per side Additionally with all playable characters being spellcasters and most enemies being melee, a combat system involving movement would end up in a pointless repetition of the player walking away and the enemies coming closer every turn. \\

\textbf{Changes compared to D\&D}
\begin{itemize}
	\item There's no Movement action.
	\item Skills involving an area, a cone et similia instead affect all enemies.
	\item Skills involving a range, line, or the concept of distance in general lose any range-related requisite.
	\item Statuses requiring a movement action to recover instead last one turn and are removed at the beginning of the next turn of the affected character. (Prone)
\end{itemize}

\fullwidthgraphicscaption{../Pictures/Gameplay/Combat_picture.png}{Game: Octopath traveler}
\pagebreak

\paragraph{Checkpoints}

Checkpoints are save points where the player can go back after a defeat. They appear as a Niffler statue which activates when the player makes an offer with a special coine called "Niffler Galleon".
\centeredgraphicssize{../Pictures/Gameplay/Statue_niffler_picture.jpg}{5cm}

\paragraph{Saving}

Checkpoints also save the game. The player can still save at anytime through the main menu. During battles saving is disabled.

\subsubsection{Rewards}

The player is rewarded upon completing battles, minigames and puzzles. Rewards can be Experience Points, money, Ingredients for Potions, Poitions, Wand Woods and Cores, House Points, and Friendship Points when Delphini is involved. (see next few sections for more details on each of those)

\centeredgraphicssize{../Pictures/Gameplay/Rewards_picture.jpg}{5cm}

\pagebreak
\paragraph{Experience Points and Friendship Points}

Characters gain Experience Points by completing challenges, missions and minigames (lessons). When Experience Points reach a certain treshold, the Character's level rises. 
Each level grants all the spells of that level which have already been seen during a lesson.

Delphini has an additional hidden stat in Frienship Points. These increase and decrease depending on the player's choices and successes in missions involving Delphini. Friendship Points are an hidden value which the player can't directly see, but the player can experience their consequences by the way Delphini interacts with Minerva and the surrounding environment.

In combat when Delphini is in the party and the friendship level is low, Delphini has access to more aggressive spells up to Forbidden Curses; additionally all her damage-dealing spells deal an extra 1d4 Necrotic Damage. In contrast, if the friendship level is high, when Minerva attacks Delphini will use a reaction action to perform an attack toghether with Minerva.

Friendship Points will ultimately determine which version of the final battle and ending the player will experience.

\paragraph{Gryffindor Points}

When the player completes a task really well (perfect score in minigames, extremely good timing in timed tasks etcc) or when he fails terribly, some points are added to or removed from the player's House Points. Sometimes player choices can affect the points of other houses as well.
Grants an achievement on the chosen game platform if Gryffindor has the highest points by the end of the game.

\pagebreak