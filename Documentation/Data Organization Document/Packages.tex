\usepackage[utf8]{inputenc}
\usepackage{graphicx}
\usepackage{placeins}
\usepackage{float}
\usepackage{titlepic}
\usepackage[T1]{fontenc}
\usepackage{babel}
\usepackage[export]{adjustbox} % sets maxwidth/maxheight for pictures
\usepackage{lipsum}
\usepackage{wrapfig} % wrap text around figures
\usepackage{caption} 
\usepackage[shortlabels]{enumitem}
\usepackage{framed}
\usepackage[margin=2cm]{geometry}
\usepackage{booktabs,tabularx}
\usepackage{titlesec}
\usepackage{longtable}

\titleformat{\section}
  {\normalfont\Large\bfseries}{\thesection}{1em}{}[{\titlerule[0.8pt]}]
\titleformat{\subsection}
  {\normalfont\Large\bfseries}{\thesubsection}{1em}{}[{\titlerule[0.5pt]}]

\newenvironment{dialogue}[1]{\begin{leftbar}#1\begin{itemize}\setlength\itemsep{0em}}{\end{itemize}\end{leftbar}}
\newcommand{\speak}[2]{\item[] \textbf{#1}: \textit{#2}}

\newcommand{\sidewrapgraphics}[2]
	{
	\begin{wrapfigure}[#2]{r}{0.5\linewidth}
	\centering
	\includegraphics[max width=0.5\textwidth]{#1}
	\end{wrapfigure}
	}
\newcommand{\sidewrapgraphicscaption}[3]
	{
	\begin{wrapfigure}[#2]{r}{0.5\linewidth}
	\centering
	\includegraphics[max width=0.5\textwidth]{#1} 
	\caption{#3}
	\end{wrapfigure}
	}
\newcommand{\fullwidthgraphics}[1]
	{
	\begin{figure}[H]
	\includegraphics[width=\textwidth]{#1} 
	\end{figure}
	}
\newcommand{\fullwidthgraphicscaption}[2]
	{
	\begin{figure}[H]
	\includegraphics[width=\textwidth]{#1} 
	\caption{#2}
	\end{figure}
	}
\newcommand{\centeredgraphics}[1]
	{
	\begin{figure}[H]
	\centering
	\includegraphics[max width=\textwidth]{#1} 
	\end{figure}
	}
\newcommand{\centeredgraphicssize}[2]
	{
	\begin{figure}[H]
	\centering
	\includegraphics[width=#2]{#1} 
	\end{figure}
	}
\newcommand{\centeredgraphicscaption}[2]
	{
	\begin{figure}[H]
	\centering
	\includegraphics[max width=\textwidth]{#1} 
	\caption{#2}
	\end{figure}
	}
	
% #1 filepath #2 img size #3 rowsamount
\newcommand{\sidewrapgraphicsleftsize}[3]
	{
	\begin{wrapfigure}[#3]{l}{#2}
	\centering
	\raisebox{0pt}[\dimexpr\height-1.2\baselineskip\relax]{\includegraphics[width=#2]{#1}}
	\end{wrapfigure}
	}
	
%%%%%%%%%%%%%%%%%%%%%%%%%%%%%%%%%%%%%%%%%%%%%%%%%%%%%%%%%%%%%%%%%%%%%%%%%%%%%%%%%%%%%%%%%%%%%%%%%%%%%%
%%%%%%%%%%%%%%%%%%%%%%%%%%%%%%% CUSTOM COMMANDS USAGE %%%%%%%%%%%%%%%%%%%%%%%%%%%%%%%%%%%%%%%%%%%%%%%%
%%%%%%%%%%%%%%%%%%%%%%%%%%%%%%%%%%%%%%%%%%%%%%%%%%%%%%%%%%%%%%%%%%%%%%%%%%%%%%%%%%%%%%%%%%%%%%%%%%%%%%
%
%
%_____________________________________________________________________________________________________
%	Text wraps around the picture. Put before the paragraph.
%	First argument is picture's path, second argument is the amount of rows the picture spans across.
%
%		\sidewrapgraphics{PATH}{ROWS TO COVER}
%_____________________________________________________________________________________________________
%	Picture which covers the whole width of the document.
%	First argument is picture's path.
%
%		\fullwidthgraphics{PATH}
%_____________________________________________________________________________________________________
%	Picture which covers the whole width of the document with a caption.
%	First argument is picture's path, second argument is picture's caption.
%
%		\fullwidthgraphicscaption{PATH}{CAPTION}
%_____________________________________________________________________________________________________
%	Picture is centered; if it's larger than the document same effect as \fullwidthgraphics
%	First argument is picture's path.
%
%		\centeredgraphics{PATH}
%_____________________________________________________________________________________________________
%	Picture is centered with caption; if it's larger than the document same effect as \fullwidthgraphics
%	First argument is picture's path, second argument is picture's caption.
%
%		\centeredgraphicscaption{PATH}{CAPTION}
%_____________________________________________________________________________________________________
%	Dialogue: wrap the dialogue in a "dialogue" environment.
%	First argument is the caption of the dialogue
%
%	\begin{dialogue}{CAPTION}
%		...
%	\end{dialogue}
%
%	Individual dialogue lines are called with path and have 2 arguments.
%	First argument is the speaker, second argument is the text.
%
%	\speak{CHARACTER}{TEXT}
%_____________________________________________________________________________________________________